%--------------------------------------------------------------------------------------------------
% OBSERVACAO:
% 
% -> Arquivos que você pode editar:
%    - artigo.tex
%    - artigo_bibliografia.bib
%
% -> Arquivo .TeX codificado em UTF8                                                             
% -> Bibliografia em arquivo .bib (arquivo_bibliografia.bib)                                      
% -> Arquivo de imagens em .jpg, .eps ou .pdf
% -> Para compilar o TeX, execute 'compila_TEX.bat' (terminal do windows)
% 
% versão 1.1 - 19/05/2016
% versão 1.0 - 18/08/2015
%--------------------------------------------------------------------------------------------------
\documentclass{classe_cn}                 % Modelo <nao edite o arquivo classe_cn.cls>
\usepackage[brazil]{babel}                % Acentos
\usepackage[utf8]{inputenc}               % Codificação UTF8 (atenção aqui!)
\usepackage{graphicx}                     % Figura
\usepackage{amssymb}                      % Simbolos matematicos
\usepackage{color}                        % Cores
\usepackage{amsfonts}                     % Fontes
\usepackage{amsmath}                      % Fontes
\usepackage[fixlanguage]{babelbib}        % Acentos
\usepackage[normalem]{ulem}               % OK
\usepackage[retainorgcmds]{IEEEtrantools} % Formulas padrão IEEE
\usepackage{omlmathbf}                    % Simbolos Matematicos
\usepackage{epstopdf}                     % Figuras .eps
\usepackage{setspace}                     % Espaçamento flexível
\usepackage{cmap}                         % Mapear caracteres especiais no PDF
\usepackage{textcomp}                     % Funções e outros símbolos matemáticos
\usepackage{verbatim}                     % Pacotes verbatim
\usepackage{wrapfig}
\usepackage{picins}
\startlocaldefs
\endlocaldefs

%--------------------------------------------------------------------------------------------------
% Inicio do Documento
%--------------------------------------------------------------------------------------------------
\begin{document}
\begin{frontmatter}        % Não alterar
\begin{fmbox}              % Não alterar
\dochead{Gerência da Informação} % Não alterar

%--------------------------------------------------------------------------------------------------
% Titulo do seu Trabalho
%   - pequeno bug (nao funciona cedilha)
%   - editar manualmente o cedilha na classe_cn.cls, linha 1015.
%--------------------------------------------------------------------------------------------------
\title{Software Livre para Empresas}

%------------------------------------------------
% Informações sobre o autor #1
% - Antunes Dantas da Silva
%------------------------------------------------
\author[
  addressref = {aff1},                 % Identifica o autor #1
  email      = {antunes.dantas@ccc.ufcg.edu.br} % email para contato
]
{
  \inits{ADdS}      % Letras iniciais do autor #1
  \fnm{Antunes Dantas}  % Nome do autor #1 (first and middle name)
  \snm{da Silva}   % Ultimo nome do autor #1 (last name)
}
%------------------------------------------------
% Informações sobre o autor #2
% - Gabriel Silva Vinha
%------------------------------------------------
\author[
  addressref = {aff1},                      % Identifica o autor
  email      = {gabriel.vinha@ccc.ufcg.edu.br} % email para contato
]
{
  \inits{GSV}       % Letras iniciais do autor #2
  \fnm{Gabriel Silva}  % Nome do autor #2 (first and middle name)
  \snm{Vinha}    % Ultimo nome do autor #2 (last name)
}
%------------------------------------------------
% Informações sobre o autor #3
% - Italo M. de L. Poroca
%------------------------------------------------
\author[
  addressref = {aff1},                       % Identifica o autor
  email      = {italo.poroca@ccc.ufcg.edu.br} % email para contato
]
{
  \inits{IMdLP}      % Letras iniciais do autor #3
  \fnm{Italo M. de Lima} % Nome do autor #3 (first and middle name)
  \snm{Poroca}    % Ultimo nome do autor #3 (last name)
}
%------------------------------------------------
% Informações sobre o autor #4
% - Valter V. M. de Lucena
%------------------------------------------------
\author[
  addressref = {aff1},                 % Identifica o autor
  email      = {valter.lucena@ccc.ufcg.edu.br} % email para contato
]
{
  \inits{VVMdL}     % Letras iniciais do autor #4
  \fnm{Valter V. M.} % Nome do autor #4 (first and middle name)
  \snm{de Lucena}     % Ultimo nome do autor #4 (last name)
}

%------------------------------------------------
% Endereço dos autores
%------------------------------------------------
\address[id=aff1]{
  \orgname{Universidade Federal de Campina Grande,
           Centro de Engenharia Elétrica e Informática,
           Departamento de Sistemas e Computação},
  \street{Rua Aprígio Veloso, 882, Bairro Universitário},
  \postcode{58429-140},
  \city{Campina Grande},
  \cny{Brasil.}
}

\end{fmbox}

%--------------------------------------------------------------------------------------------------
% Resumo do Trabalho
%--------------------------------------------------------------------------------------------------
\begin{abstractbox}
	
\begin{abstract} 
Escrever no máximo $150$ palavras no resumo do trabalho. Exemplo: The objective of this work is to determine if people are interacting in TV video by detecting whether they are looking at each other or not.We determine both the temporal period of the interaction and also spatially localize the relevant people. We make the following four contributions: (\textit{i}) head detection with implicit coarse pose information (front, profile, back); (\textit{ii}) continuous head pose estimation in unconstrained scenarios (TV video) using Gaussian process regression; (\textit{iii}) propose and evaluate several methods for assessing whether and when pairs of people are looking at each other in a video shot; and (\textit{iv}) introduce new ground truth annotation for this task, extending the TV human interactions dataset. The performance of the methods is evaluated on this dataset, which consists of $300$ video clips extracted from TV shows. Despite the variety and difficulty of this video material, our best method obtains an average precision of $87.6\%$ in a fully automatic manner.
\end{abstract}

%--------------------------------------------------------------------------------------------------
% Palavras-chaves: Entre 3 e 6 palavras chaves
%--------------------------------------------------------------------------------------------------
\begin{keyword}
  \kwd{Escreva}
  \kwd{algumas}
  \kwd{palavras-chaves}
  \kwd{aqui!}
\end{keyword}

\end{abstractbox} % Não alterar
\end{frontmatter} % Não alterar

%--------------------------------------------------------------------------------------------------
% Escreva o seu artigo!
%--------------------------------------------------------------------------------------------------

%------------------------------------------------
% Seção 1
%------------------------------------------------
\section{Introdução}

O \textit{software} livre é uma realidade que existe desde os primórdios da computação. Baseando-se na ideia básica de que o código fonte deve ser público, o movimento do \textit{software} livre gerou, e ainda gera, bastante polêmica dentre a comunidade da tecnologia da informação, especialmente quando o assunto tange as grandes corporações que lucram com a venda de \textit{softwares} proprietários. Como movimento, iniciou em 1983 com um americano chamado Richard Stallman, que liderou o desenvolvimento de um sistema operacional baseado totalmente nas ideias do \textit{software} livre.

Para ser considerado livre, um \textit{software} deve seguir determinadas "leis", que definem como ele deve ser publicado. Para facilitar a publicação, foram criadas licenças genéricas que servem para qualquer \textit{software}.

Um dos principais questionamentos quando o assunto é tratado é como empresas podem faturar fabricando código aberto. Como será exposto posteriormente, existem diversos modelos de negócios que podem ser abordados para este fim.

Este artigo seguirá a seguinte estrutura: na seção 2, será mostrada a motivação para este estudo. Na seção 3, o tema \textit{software} livre será abordado de maneira mais detalhada, bem como modalidades que onde este é encontrado. Na seção 4, será realizado um breve estudo sobre as principais licenças de publicação. Finalmente, na seção 5, será tratado como empresas podem fazer o uso de \textit{software} livre: tanto no lado cliente quanto no lado empresas produtoras. A seção 6 fará uma discussão sobre o futuro da distribuição dos \textit{softwares} e como o \textit{software} livre se encaixa nessa realidade futura.


%------------------------------------------------
% Seção 2
%------------------------------------------------
\section{Motivação}

If we assume that sensitive cells follow a deterministic decay $Z_0(t) = xe^{\lambda_0 t}$ and approximate their extinction time as $T_x \approx \frac{1}{\lambda_0} \log x$, then we can heuristically estimate the expected value as:

\begin{eqnarray}
\label{eqexpmuts}
  E [Z_1(vT_x)] &=& \frac{\mu}{r}\log x \int_0^{1} x^{1-u} du \\
  E [Z_1(vT_x)] &=& \frac{\mu}{r}x^{1-{\lambda_1}/{\lambda_0}v}\log  \\
  1 &=& 10
\end{eqnarray}

\begin{equation}
  E [Z_1(vT_x)] = \frac{\mu}{r}\log x \int_0^{1} x^{1-u} du \\
  E [Z_1(vT_x)] = \frac{\mu}{r}x^{1-{\lambda_1}/{\lambda_0}v}\log 
\end{equation}

Thus we observe that this expected value is finite for all $v>0$ (also see \cite{Rosenfeld:1970}).

%------------------------------------------------
% Sub-seção
%------------------------------------------------
\subsection{Exemplo de Sub-Seção}

In this section we examine the growth rate of the mean of $Z_0$, $Z_1$ and $Z_2$. In addition, we examine a common modeling assumption and note the importance of considering the tails of the extinction time $T_x$ in studies of escape dynamics. We will first consider the expected resistant population at $vT_x$ for some $v>0$, (and temporarily assume $\alpha=0$).

\begin{eqnarray}
E [Z_1(vT_x)]= \mu T_x \int_{0}^{\inf} \lambda_1T_x(v-u)du
\end{eqnarray}

If we assume that sensitive cells follow a deterministic decay $Z_0(t)=xe^{\lambda_0 t}$ and approximate their extinction time as $T_x\approx-\frac{1}{\lambda_0}\log x$, then we can heuristically estimate the expected value as.

%------------------------------------------------
% Exemplo de Tabela
%------------------------------------------------
\section{Software Livre}

Table~\ref{tag_tabela_01} shows the average $ \alpha $ and the standard deviation for the CCR \cite{Rosenfeld:1970} obtained by the \textit{GLCM+SOM} method. We can conclude that for the Brodatz dataset~\cite{Domingues:2010} the processing tool based on mean vectors is the best option~\cite{Rosenfeld:1970, Diday:1989}. Considering this result~\cite{Visible:2013}, the mean vector approach is adopted as processing tool of the \textit{GLCM+SOM} method for the next experiments \cite{Fulano:2009}.

testando 123

% Use a ferramenta para criar tabelas: http://www.tablesgenerator.com/
\begin{table}[h!]
\label{tag_tabela_01}
\caption{Sample table title. This is where the description of the table should go.}
  \begin{tabular}{cccc}
  \hline
       & B1   & B2   & B3   \\ \hline
   A1  & 0.1  & 0.2  & 0.3  \\
   A2  & ...  & ..   & .    \\
   A3  & ..   & .    & .    \\ \hline
  \end{tabular}
\end{table}

\subsection{Software as a Product}

aqui vc faz

\subsection{Software as a Service}

O Software como Serviço, do inglês \textit{Software as a Service} (SaaS), é o nome que se dá ao modo de comercialização de \textit{software} que não se baseia na venda de licenças de uso, e sim na cobrança de uma taxa regular para sua utilização. Diferente da forma tradicional de distribuição de \textit{software}, onde os usuários adquirem uma licença e são responsáveis pela instalação e manutenção do sistema, essas aplicações são executadas exclusivamente nos servidores dos seus provedores, fazendo-se necessária ao usuário apenas uma conexão com a Internet para utilizá-las.

Neste modelo, os fornecedores são inteiramente responsáveis pela manutenção das estruturas necessárias (servidores, conexões, segurança, entre outras) para a disponibilização do sistema, e o usuário utiliza o serviço através da Internet mediante pagamento de uma certa taxa recorrente. 

Existem diversas formas de se cobrar por estes serviços. As mais comuns variam entre cobrar um valor (na maioria das vezes mensal) proporcional ao número de licenças utilizadas, cobrar um valor proporcional ao uso, e modelos \textit{freemium} que oferecem serviços gratuitos mas cobram por recursos adicionais.

Um exemplo de SaaS é o OpenStack, oferecido pela empresa RedHat, que está para a computação em nuvem assim como o sistema operacional está para os computadores. Trata-se de um conjunto de softwares que possibilitam a criação de ferramentas para o gerenciamento de nuvens públicas ou privadas, planejado numa parceria entre a NASA e a RacksPace Hosting com o objetivo de que as empresas consumissem serviços em nuvem com elementos de \textit{hardware} padrão.

\subsection{Componentes da Produção de Software}

De acordo com a definição criada pela Free Software Foundation, \textit{software} livre é aquele que pode ser usado, modificado, estudado, copiado e redistribuído sem restrições. Para um usuário leigo, isso talvez não signifique muita coisa, mas para um usuário desenvolvedor, lhe dá a liberdade de modificar o \textit{software} livremente para que melhor se adapte as suas necessidades, ou para redistribuí-lo de alguma forma. É o que acontece com o sistema operacional Ubuntu, patrocinado pela empresa Cannonical Ltd., que permite que seus usuarios contribuam para o desenvolvimento do sistema.

A liberdade de modificar ou redistribuir o \textit{software} implica na necessidade de acesso ao código-fonte, mas para que essa liberdade seja real, essas modificações devem acontecer sem que o distribuidor original possa refutá-las sem algum motivo. Caso isso aconteça, o software não é livre.

Para isso, foram criadas as licenças de \textit{software}, que consistem nas ações que um usuário pode ou não executar em relação a um determinado software. 

\section{Licenças de Publicação}

tarara

\section{Software Livre Para Empresas}


\subsection{estatisticas de mercado para saap}
oi


\subsection{Software as a service}

aqui vc faz

\subsection{core}

\section{Tendências}

olar


%--------------------------------------------------------------------------------------------------
%--------------------------------------------------------------------------------------------------
% Define o arquivo BIB (bibliografia)
%--------------------------------------------------------------------------------------------------
%--------------------------------------------------------------------------------------------------
\bibliographystyle{bmc-mathphys}   % NAO EDITAR!
\bibliography{artigo_bibliografia} % NAO EDITAR! - Bibliography file (usually '*.bib' )

\vspace{1.0cm}
\parpic{\includegraphics[width=1.5in,clip,keepaspectratio]{tesla.jpg}}
\noindent {\bf Fulano de Tal} was born in India. She received the B.S. 
degree in computer science from Kurukshetra University, Kurukshetra, 
India and the M.Phil. and Ph.D. degrees from the University of Exeter, 
Exeter, UK in 1999, 2001 and 2004, respectively. Her Ph.D. was in the 
area of machine learning for image analysis in aviation security. Her 
main research interests include image processing, natural scene analysis,
video analysis, and neural networks. She has published more than 30 papers
in the area of machine learning for image analysis in peer reviewed 
journals and conferences. Currently she is a Senior Research Fellow at
Loughborough University leading the project on imaging for road transport
applications.

\parpic{\includegraphics[width=1.5in,clip,keepaspectratio]{tesla.jpg}}
\noindent {\bf Fulano de Tal} was born in India. She received the B.S. 
degree in computer science from Kurukshetra University, Kurukshetra, 
India and the M.Phil. and Ph.D. degrees from the University of Exeter, 
Exeter, UK in 1999, 2001 and 2004, respectively. Her Ph.D. was in the 
area of machine learning for image analysis in aviation security. Her 
main research interests include image processing, natural scene analysis,
video analysis, and neural networks. She has published more than 30 papers
in the area of machine learning for image analysis in peer reviewed 
journals and conferences. Currently she is a Senior Research Fellow at
Loughborough University leading the project on imaging for road transport
applications.   


%\end{tabular}
%\end{table}

%--------------------------------------------------------------------------------------------------
% FIM DO ARTIGO
%--------------------------------------------------------------------------------------------------
\end{document}
